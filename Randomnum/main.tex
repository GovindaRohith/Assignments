\documentclass[journal,12pt,twocolumn]{IEEEtran}
\usepackage{setspace}
\usepackage{gensymb}
\usepackage{caption}
%\usepackage{multirow}
%\usepackage{multicolumn}
%\usepackage{subcaption}
%\doublespacing
\singlespacing
\usepackage{csvsimple}
\usepackage{amsmath}
\usepackage{multicol}
%\usepackage{enumerate}
\usepackage{amssymb}
%\usepackage{graphicx}
\usepackage{newfloat}
%\usepackage{syntax}
\usepackage{listings}
\usepackage{color}
\usepackage{tikz}
\usetikzlibrary{shapes,arrows}



%\usepackage{graphicx}
%\usepackage{amssymb}
%\usepackage{relsize}
%\usepackage[cmex10]{amsmath}
%\usepackage{mathtools}
%\usepackage{amsthm}
%\interdisplaylinepenalty=2500
%\savesymbol{iint}
%\usepackage{txfonts}
%\restoresymbol{TXF}{iint}
%\usepackage{wasysym}
\usepackage{amsthm}
\usepackage{mathrsfs}
\usepackage{txfonts}
\usepackage{stfloats}
\usepackage{cite}
\usepackage{cases}
\usepackage{mathtools}
\usepackage{caption}
\usepackage{enumerate}	
\usepackage{enumitem}
\usepackage{amsmath}
%\usepackage{xtab}
\usepackage{longtable}
\usepackage{multirow}
%\usepackage{algorithm}
%\usepackage{algpseudocode}
\usepackage{enumitem}
\usepackage{mathtools}
\usepackage{hyperref}
%\usepackage[framemethod=tikz]{mdframed}
\usepackage{listings}
    %\usepackage[latin1]{inputenc}                                 %%
    \usepackage{color}                                            %%
    \usepackage{array}                                            %%
    \usepackage{longtable}                                        %%
    \usepackage{calc}                                             %%
    \usepackage{multirow}                                         %%
    \usepackage{hhline}                                           %%
    \usepackage{ifthen}                                           %%
  %optionally (for landscape tables embedded in another document): %%
    \usepackage{lscape}     


\usepackage{url}
\def\UrlBreaks{\do\/\do-}


%\usepackage{stmaryrd}


%\usepackage{wasysym}
%\newcounter{MYtempeqncnt}
\DeclareMathOperator*{\Res}{Res}
%\renewcommand{\baselinestretch}{2}
\renewcommand\thesection{\arabic{section}}
\renewcommand\thesubsection{\thesection.\arabic{subsection}}
\renewcommand\thesubsubsection{\thesubsection.\arabic{subsubsection}}

\renewcommand\thesectiondis{\arabic{section}}
\renewcommand\thesubsectiondis{\thesectiondis.\arabic{subsection}}
\renewcommand\thesubsubsectiondis{\thesubsectiondis.\arabic{subsubsection}}

% correct bad hyphenation here
\hyphenation{op-tical net-works semi-conduc-tor}

%\lstset{
%language=C,
%frame=single, 
%breaklines=true
%}

%\lstset{
	%%basicstyle=\small\ttfamily\bfseries,
	%%numberstyle=\small\ttfamily,
	%language=Octave,
	%backgroundcolor=\color{white},
	%%frame=single,
	%%keywordstyle=\bfseries,
	%%breaklines=true,
	%%showstringspaces=false,
	%%xleftmargin=-10mm,
	%%aboveskip=-1mm,
	%%belowskip=0mm
%}

%\surroundwithmdframed[width=\columnwidth]{lstlisting}
\def\inputGnumericTable{}                                 %%
\lstset{
%language=C,
frame=single, 
breaklines=true,
columns=fullflexible
}
 

\begin{document}
%
\tikzstyle{block} = [rectangle, draw,
    text width=3em, text centered, minimum height=3em]
\tikzstyle{sum} = [draw, circle, node distance=3cm]
\tikzstyle{input} = [coordinate]
\tikzstyle{output} = [coordinate]
\tikzstyle{pinstyle} = [pin edge={to-,thin,black}]
\providecommand{\e}[1]{\ensuremath{E\left(#1\right)}}
\providecommand{\es}[1]{\ensuremath{E\left[#1\right]}}
\theoremstyle{definition}
\newtheorem{theorem}{Theorem}[section]
\newtheorem{problem}{Problem}
\newtheorem{proposition}{Proposition}[section]
\newtheorem{lemma}{Lemma}[section]
\newtheorem{corollary}[theorem]{Corollary}
\newtheorem{example}{Example}[section]
\newtheorem{definition}{Definition}[section]
%\newtheorem{algorithm}{Algorithm}[section]
%\newtheorem{cor}{Corollary}
\newcommand{\BEQA}{\begin{eqnarray}}
\newcommand{\EEQA}{\end{eqnarray}}
\newcommand{\define}{\stackrel{\triangle}{=}}
\bibliographystyle{IEEEtran}
%\bibliographystyle{ieeetr}
\providecommand{\nCr}[2]{\,^{#1}C_{#2}} % nCr
\providecommand{\nPr}[2]{\,^{#1}P_{#2}} % nPr
\providecommand{\mbf}{\mathbf}
\providecommand{\pr}[1]{\ensuremath{\Pr\left(#1\right)}}
\providecommand{\qfunc}[1]{\ensuremath{Q\left(#1\right)}}
\providecommand{\sbrak}[1]{\ensuremath{{}\left[#1\right]}}
\providecommand{\lsbrak}[1]{\ensuremath{{}\left[#1\right.}}
\providecommand{\rsbrak}[1]{\ensuremath{{}\left.#1\right]}}
\providecommand{\gauss}[2]{\mathcal{N}\ensuremath{\left(#1,#2\right)}}
\providecommand{\brak}[1]{\ensuremath{\left(#1\right)}}
\providecommand{\lbrak}[1]{\ensuremath{\left(#1\right.}}
\providecommand{\rbrak}[1]{\ensuremath{\left.#1\right)}}
\providecommand{\cbrak}[1]{\ensuremath{\left\{#1\right\}}}
\providecommand{\lcbrak}[1]{\ensuremath{\left\{#1\right.}}
\providecommand{\rcbrak}[1]{\ensuremath{\left.#1\right\}}}
\theoremstyle{remark}
\newtheorem{rem}{Remark}
\newcommand{\sgn}{\mathop{\mathrm{sgn}}}
\providecommand{\abs}[1]{\left\vert#1\right\vert}
\providecommand{\res}[1]{\Res\displaylimits_{#1}} 
\providecommand{\norm}[1]{\left\Vert#1\right\Vert}
\providecommand{\mtx}[1]{\mathbf{#1}}
\providecommand{\mean}[1]{E\left[ #1 \right]}
\providecommand{\fourier}{\overset{\mathcal{F}}{ \rightleftharpoons}}
%\providecommand{\hilbert}{\overset{\mathcal{H}}{ \rightleftharpoons}}
\providecommand{\system}{\overset{\mathcal{H}}{ \longleftrightarrow}}
	%\newcommand{\solution}[2]{\textbf{Solution:}{#1}}
\newcommand{\solution}{\noindent \textbf{Solution: }}
\newcommand{\myvec}[1]{\ensuremath{\begin{pmatrix}#1\end{pmatrix}}}
\newcommand{\mydet}[1]{\ensuremath{\begin{vmatrix}#1\end{vmatrix}}}
\providecommand{\dec}[2]{\ensuremath{\overset{#1}{\underset{#2}{\gtrless}}}}
\DeclarePairedDelimiter{\ceil}{\lceil}{\rceil}
%\numberwithin{equation}{section}
%\numberwithin{problem}{subsection}
%\numberwithin{definition}{subsection}
\makeatletter
\@addtoreset{figure}{section}
\makeatother
\let\StandardTheFigure\thefigure
%\renewcommand{\thefigure}{\theproblem.\arabic{figure}}
\renewcommand{\thefigure}{\thesection}
%\numberwithin{figure}{subsection}
%\numberwithin{equation}{subsection}
%\numberwithin{equation}{section}
%\numberwithin{equation}{problem}
%\numberwithin{problem}{subsection}
\numberwithin{problem}{section}
%%\numberwithin{definition}{subsection}
%\makeatletter
%\@addtoreset{figure}{problem}
%\makeatother
\makeatletter
\@addtoreset{table}{section}
\makeatother
\let\StandardTheFigure\thefigure
\let\StandardTheTable\thetable
\let\vec\mathbf
\numberwithin{equation}{section}
\vspace{3cm}
\title{%Convex Optimization in Python
	{
	Random Numbers
	}
}
%\title{
%	\logo{Matrix Analysis through Octave}{\begin{center}\includegraphics[scale=.24]{tlc}\end{center}}{}{HAMDSP}
%}
% paper title
% can use linebreaks \\ within to get better formatting as desired
%\title{Matrix Analysis through Octave}
%
%
% author names and IEEE memberships
% note positions of commas and nonbreaking spaces ( ~ ) LaTeX will not break
% a structure at a ~ so this keeps an author's name from being broken across
% two lines.
% use \thanks{} to gain access to the first footnote area
% a separate \thanks must be used for each paragraph as LaTeX2e's \thanks
% was not built to handle multiple paragraphs
%
\author{ *Govinda Rohith Y}
% note the % following the last \IEEEmembership and also \thanks - 
% these prevent an unwanted space from occurring between the last author name
% and the end of the author line. i.e., if you had this:
% 
% \author{....lastname \thanks{...} \thanks{...} }
%                     ^------------^------------^----Do not want these spaces!
%
% a space would be appended to the last name and could cause every name on that
% line to be shifted left slightly. This is one of those "LaTeX things". For
% instance, "\textbf{A} \textbf{B}" will typeset as "A B" not "AB". To get
% "AB" then you have to do: "\textbf{A}\textbf{B}"
% \thanks is no different in this regard, so shield the last } of each \thanks
% that ends a line with a % and do not let a space in before the next \thanks.
% Spaces after \IEEEmembership other than the last one are OK (and needed) as
% you are supposed to have spaces between the names. For what it is worth,
% this is a minor point as most people would not even notice if the said evil
% space somehow managed to creep in.
% The paper headers
%\markboth{Journal of \LaTeX\ Class Files,~Vol.~6, No.~1, January~2007}%
%{Shell \MakeLowercase{\textit{et al.}}: Bare Demo of IEEEtran.cls for Journals}
% The only time the second header will appear is for the odd numbered pages
% after the title page when using the twoside option.
% 
% *** Note that you probably will NOT want to include the author's ***
% *** name in the headers of peer review papers.                   ***
% You can use \ifCLASSOPTIONpeerreview for conditional compilation here if
% you desire.
% If you want to put a publisher's ID mark on the page you can do it like
% this:
%\IEEEpubid{0000--0000/00\$00.00~\copyright~2007 IEEE}
% Remember, if you use this you must call \IEEEpubidadjcol in the second
% column for its text to clear the IEEEpubid mark.
% make the title area
\maketitle
\tableofcontents
\bigskip
\renewcommand{\thefigure}{\theenumi}
\renewcommand{\thetable}{\theenumi}
\begin{abstract}
This manual provides solutions to the Assignment on Random Numbers
\end{abstract}
%template ends here
\section{Uniform Random Numbers}
Let $U$ be a uniform random variable between 0 and 1.
\begin{enumerate}[label=\thesection.\arabic*
,ref=\thesection.\theenumi]
\item Generate $10^6$ samples of $U$ using a C program and save into a file called uni.dat .
\\
\solution Download the following files and execute the  C program.
\begin{lstlisting}
wget https://github.com/GovindaRohith/Assignments/blob/main/Randomnum/codes/1.1.c
wget https://github.com/GovindaRohith/Assignments/blob/main/Randomnum/codes/source.h
\end{lstlisting}
Download the above files and execute the following commands
\begin{lstlisting}
$ gcc 1.1.c
$ ./a.out
\end{lstlisting}
\item
Load the uni.dat file into python and plot the empirical CDF of $U$ using the samples in uni.dat. The CDF is defined as
\begin{align}
F_{U}(x) = \pr{U \le x}
\end{align}
\\
\solution  The following code plots Fig. \ref{fig:1.2}
\begin{lstlisting}
wget https://github.com/GovindaRohith/Assignments/blob/main/Randomnum/codes/1.2.py
\end{lstlisting}
Download the above files and execute the following commands to produce Fig.\ref{fig:1.2}
\begin{lstlisting}
$ python3 1.2.py
\end{lstlisting}
\begin{figure}[!h]
\centering
\includegraphics[width=\columnwidth]{./figs/1.2.png}
\caption{The CDF of $U$}
\label{fig:1.2}
\end{figure}

%
\item
Find a  theoretical expression for $F_{U}(x)$.\\
\solution Given $U$ is a uniform Random Variable
\begin{align}
p_{U}(x)=1 \text{ for } \\
F_U(x)=\int_{-\infty}^{\infty}p_{U}(x)dx\\
\boxed{\implies F_U(x)=
\begin{cases}
 0 &x\le0\\
 x &0< x< 1\\
 1 &x\ge 1
\end{cases}}
\end{align}
\item
The mean of $U$ is defined as
%
\begin{equation}
E\sbrak{U} = \frac{1}{N}\sum_{i=1}^{N}U_i
\end{equation}
%
and its variance as
%
\begin{equation}
\text{var}\sbrak{U} = E\sbrak{U- E\sbrak{U}}^2 
\end{equation}

Write a C program to  find the mean and variance of $U$. \\
\solution Download the following files and execute the  C program.
\begin{lstlisting}
wget https://github.com/GovindaRohith/Assignments/blob/main/Randomnum/codes/1.4.c
wget https://github.com/GovindaRohith/Assignments/blob/main/Randomnum/codes/source.h
\end{lstlisting}
Download the above files and execute the following commands
\begin{lstlisting}
$ gcc 1.4.c
$ ./a.out
\end{lstlisting}
\item Verify your result theoretically given that
\end{enumerate}
%
\begin{equation}
E\sbrak{U^k} = \int_{-\infty}^{\infty}x^kdF_{U}(x)
\end{equation}
\solution 
\begin{align}
    \text{var}\sbrak{U} &= E\sbrak{U- E\sbrak{U}}^2\\ 
    \implies \text{var}\sbrak{U} &= E\sbrak{U^2}- E\sbrak{U}^2 \\
    E\sbrak{U}&=\int_{-\infty}^{\infty}xdF_U(x)\\
    E\sbrak{U}&=\int_{0}^{1}x\\
    \implies \boxed{E\sbrak{U}=\frac{1}{2}}\\
    E\sbrak{U^2}&=\int_{-\infty}^{\infty}x^{2}dF_U(x)\\
    E\sbrak{U^2}&=\int_{0}^{1}x^{2}dF_U(x)\\
    \implies E\sbrak{U^2}&=\frac{1}{3}\\
    \implies \boxed{\text{var}\sbrak{U}=\frac{1}{12}=0.0833}
\end{align}
\section{Central Limit Theorem}
%
\begin{enumerate}[label=\thesection.\arabic*
,ref=\thesection.\theenumi]

%
\item
Generate $10^6$ samples of the random variable
%
\begin{equation}
X = \sum_{i=1}^{12}U_i -6
\end{equation}
%
using a C program, where $U_i, i = 1,2,\dots, 12$ are  a set of independent uniform random variables between 0 and 1 and save in a file called gau.dat\\
\solution Download the following files and execute the  C program.
\begin{lstlisting}
wget https://github.com/GovindaRohith/Assignments/blob/main/Randomnum/codes/2.1.c
wget https://github.com/GovindaRohith/Assignments/blob/main/Randomnum/codes/source.h
\end{lstlisting}
Download the above files and execute the following commands
\begin{lstlisting}
$ gcc 2.1.c
$ ./a.out
\end{lstlisting}
\item
Load gau.dat in python and plot the empirical CDF of $X$ using the samples in gau.dat. What properties does a CDF have?\\
\solution The CDF of $X$ is plotted in Fig. \ref{fig:2.2}\\
using the code below
\begin{lstlisting}
wget https://github.com/GovindaRohith/Assignments/blob/main/Randomnum/codes/2.2.py
\end{lstlisting}
Download the above files and execute the following commands to produce Fig.\ref{fig:2.2}
\begin{lstlisting}
$ python3 2.2.py
\end{lstlisting}
\begin{figure}[!h]
\centering
\includegraphics[width=\columnwidth]{./figs/2.2.png}
\caption{The CDF of $X$}
\label{fig:2.2}
\end{figure}
Some of the properties of CDF 
\begin{enumerate}
\item $\lim_{x \to \infty}F_X(x) = 1$
    \item $F_X(x)$ is non decreasing function.
    \item Symmetric about one point.
\end{enumerate}
\item
Load gau.dat in python and plot the empirical PDF of $X$ using the samples in gau.dat. The PDF of $X$ is defined as
\begin{align}
p_{X}(x) = \frac{d}{dx}F_{X}(x)
\end{align}
What properties does the PDF have?
\\
\solution The PDF of $X$ is plotted in Fig. \ref{fig:2.3} using the code below
\begin{lstlisting}
wget https://github.com/GovindaRohith/Assignments/blob/main/Randomnum/codes/2.3.py
\end{lstlisting}
Download the above files and execute the following commands to produce Fig.\ref{fig:2.3}
\begin{lstlisting}
$ python3 2.3.py
\end{lstlisting}
\begin{figure}[!h]
\centering
\includegraphics[width=\columnwidth]{./figs/2.3.png}
\caption{The PDF of $X$}
\label{fig:2.3}
\end{figure}
Some of the properties of the PDF:
\begin{enumerate}
    \item Symmetric about $x=\mu$ in this case
    \item Decreasing function for $x>\mu$ and increasing for $x<\mu$ and attains maximum at $x=\mu$
    \item Area under the curve is unity.
\end{enumerate}
\item Find the mean and variance of $X$ by writing a C program.\\
\solution Download the following files and execute the  C program.
\begin{lstlisting}
wget https://github.com/GovindaRohith/Assignments/blob/main/Randomnum/codes/2.4.c
wget https://github.com/GovindaRohith/Assignments/blob/main/Randomnum/codes/source.h
\end{lstlisting}
Download the above files and execute the following commands
\begin{lstlisting}
$ gcc 2.4.c
$ ./a.out
\end{lstlisting}
\item Given that 
\begin{align}
p_{X}(x) = \frac{1}{\sqrt{2\pi}}\exp\brak{-\frac{x^2}{2}}, -\infty < x < \infty,
\end{align}
repeat the above exercise theoretically.
\end{enumerate}
\solution 
\begin{enumerate}
    \item CDF is given by 
    \begin{align}
        F_X(x)&=\int_{-\infty}^{\infty}p_X(x)dx\\
        \boxed{F_X(x)=1}
    \end{align}
    \item Mean is given by
    \begin{align}
        E(x)=\int_{-\infty}^{\infty}xp_X(x)dx\\
        \implies \boxed{E(x)=0}
    \end{align}
    \item Variance is given by
    \begin{align}
        \text{var}\sbrak{U}&=E(U^2)-(E(U))^2\\
E{x^2}&=\int_{-\infty}^{\infty}x^2p_X(x)dx \\
&=\int_{-\infty}^{\infty}\frac{1}{\sqrt{2\pi}}x^2exp\brak{-\frac{x^2}{2}}dx \\
&=\frac{1}{\sqrt{2\pi}}\brak{x\int x exp\brak{-\frac{x^2}{2}}dx}
\\ &-\frac{1}{\sqrt{2\pi}}\int \int \brak{x exp\brak{-\frac{x^2}{2}}}dx. dx\\
&=\frac{1}{\sqrt{2\pi}}\int_{-\infty}^{\infty}exp\brak{-\frac{x^2}{2}}dx \\
&=\frac{\sqrt{2\pi}}{\sqrt{2\pi}}\\&=1
        \implies\boxed{\text{var}\sbrak{U}=1}
    \end{align}
\end{enumerate}
\section{From Uniform to Other}
\begin{enumerate}[label=\thesection.\arabic*
,ref=\thesection.\theenumi]
%
\item
Generate samples of 
%
\begin{equation}
V = -2\ln\brak{1-U}
\end{equation}
%
and plot its CDF.  \\
\solution Download the following files and execute the  C program.
\begin{lstlisting}
wget https://github.com/GovindaRohith/Assignments/blob/main/Randomnum/codes/3.1.c
wget https://github.com/GovindaRohith/Assignments/blob/main/Randomnum/codes/source.h
\end{lstlisting}
Download the above files and execute the following commands
\begin{lstlisting}
$ gcc 3.1.c -lm
$ ./a.out
\end{lstlisting}
The CDF of $V$ is plotted in Fig. \ref{fig:3.1} using the code below
\begin{lstlisting}
wget https://github.com/GovindaRohith/Assignments/blob/main/Randomnum/codes/3.1pyth.py
\end{lstlisting}
Download the above files and execute the following commands to produce Fig.\ref{fig:3.1}
\begin{lstlisting}
$ python3 3.1pyth.py
\end{lstlisting}
\begin{figure}[!h]
\centering
\includegraphics[width=\columnwidth]{./figs/3.1.png}
\caption{The CDF of $V$}
\label{fig:3.1}
\end{figure}
\item Find a theoretical expression for $F_V(x)$.\\
\solution
If Y = g(X), we know that $F_Y(y) = F_X(g^{-1}(y))$, here 
\begin{align}
V &= -2\ln{(1-U)} \\
1-U &= e^{\frac{-V}{2}}\\
U &= 1 - e^{\frac{-V}{2}} \\ 
F_V(x) &= F_U(1 - e^{\frac{-x}{2}}) 
\end{align}
 \begin{align}
\implies
  F_V(x)=
  \begin{cases}
   0                         & x < 0 \\
	1 - e^{\frac{-x}{2}} & x \geq 0
	\end{cases}
 \end{align}
%\item
%Generate the Rayleigh distribution from Uniform. Verify your result through graphical plots.
\end{enumerate}
\section{Triangular Distribution}
\begin{enumerate}[label=\thesection.\arabic*
,ref=\thesection.\theenumi]
    \item Generate
    \begin{align}
        T=U_1+U_2
    \end{align}
    \solution Download the following files and execute the  C program.
\begin{lstlisting}
wget https://github.com/GovindaRohith/Assignments/blob/main/Randomnum/codes/4.1.c
wget https://github.com/GovindaRohith/Assignments/blob/main/Randomnum/codes/source.h
\end{lstlisting}
Download the above files and execute the following commands
\begin{lstlisting}
$ gcc 4.1.c
$ ./a.out
\end{lstlisting}
\item Find the CDF of $T$.\\
\solution The CDF of $T$ is plotted in Fig. \ref{fig:4.2} using the code below
\begin{lstlisting}
wget https://github.com/GovindaRohith/Assignments/blob/main/Randomnum/codes/4.5cdf.py
\end{lstlisting}
Download the above files and execute the following commands to produce Fig.\ref{fig:4.2}
\begin{lstlisting}
$ python3 4.5cdf.py
\end{lstlisting}
\begin{figure}[!h]
\centering
\includegraphics[width=\columnwidth]{./figs/4.5cdf.png}
\caption{The CDF of $T$}
\label{fig:4.2}
\end{figure}
\item Find the PDF of $T$.\\
\solution The PDF of $T$ is plotted in Fig. \ref{fig:4.2} using the code below
\begin{lstlisting}
wget https://github.com/GovindaRohith/Assignments/blob/main/Randomnum/codes/4.5pdf.py
\end{lstlisting}
Download the above files and execute the following commands to produce Fig.\ref{fig:4.2}
\begin{lstlisting}
$ python3 4.5pdf.py
\end{lstlisting}
\begin{figure}[!h]
\centering
\includegraphics[width=\columnwidth]{./figs/4.5pdf.png}
\caption{The PDF of $T$}
\label{fig:4.3}
\end{figure}
\item Find the Theoreotical Expression for the PDF and CDF of $T$\\
\solution
\begin{align}
    T&=U_1+U_2\\
    \implies p_T(t)&=\int_{-\infty}^{t}p_{U1}(x)p_{U2}(y)dx\\
    \text{As,}p_{U1}(x)&=p_{U1}(y)=p_{U}(u)\\
    \implies p_T(t)&=\int_{-\infty}^{t}p_{U}(u)p_{U}(t-u)du
    \end{align}
    \begin{enumerate}
        \item Theoretical PDF 
        \begin{enumerate}
            \item $t\le 1$
            \begin{align}
                p_T(t)&=\int_{0}^{t}p_{U}(t-u)du\\
                \implies p_T(t)&=\int_{0}^{t} du=t
            \end{align}
            \item $t> 1$
             \begin{align}
                p_T(t)&=\int_{0}^{1}p_{U}(t-u)du\\
                \implies p_T(t)&=\int_{t-1}^{1} du=2-t
            \end{align}
        \end{enumerate}
        $\implies\boxed{ P_T(t) =
        \begin{cases}
         t     &0 \le t \le 1 \\
         2-t   &1 < t \le 2\\
         0     &t<0 \text{ or }t>2
        \end{cases}
        }$\\
        \item Theoretical CDF 
        \begin{align}
            F_T(t)=\int_{-\infty}^{t}p_T(u)du
        \end{align}
        $\implies\boxed{
            F_{T}(t)=
            \begin{cases}
             0   &t<0\\
             \dfrac{t^2}{2} &0\le t \le 1\\
             2t-1-\dfrac{t^2}{2} &1<t \le 2\\
             1 &t>2
            \end{cases}
            }$\\
    \end{enumerate}
\item Verify your results through a plot \\
\solution The Results are verfied in the plots
Fig \ref{fig:4.2} and Fig \ref{fig:4.3}
\section{Guasssian to Other}
\begin{enumerate}[label=\thesection.\arabic*
,ref=\thesection.\theenumi]
\item Generate equiprobable $X \in \cbrak{1,-1}$.
\solution Download the following files and execute the  C program.
\begin{lstlisting}
wget https://github.com/GovindaRohith/Assignments/blob/main/Randomnum/codes/5.1.c
wget https://github.com/GovindaRohith/Assignments/blob/main/Randomnum/codes/source.h
\end{lstlisting}
Download the above files and execute the following commands
\begin{lstlisting}
$ gcc 5.1.c
$ ./a.out
\end{lstlisting}
\item Generate
\begin{align}
    Y=AX+N,
\end{align}
Where A=5 dB and $N\sim \gauss{0}{1}$\\
\solution:
\begin{lstlisting}
wget https://github.com/GovindaRohith/Assignments/blob/main/Randomnum/codes/5.2.c
wget https://github.com/GovindaRohith/Assignments/blob/main/Randomnum/codes/source.h
\end{lstlisting}
Download the above files and execute the following commands
\begin{lstlisting}
$ gcc 5.2.c
$ ./a.out
\end{lstlisting}
	\item Plot $Y$ using a scatter plot.\\
	\solution
	The CDF of $V$ is plotted in Fig. \ref{fig:5.3} using the code below
\begin{lstlisting}
wget https://github.com/GovindaRohith/Assignments/blob/main/Randomnum/codes/5.3.py
\end{lstlisting}
Download the above files and execute the following commands to produce Fig.\ref{fig:5.3}
\begin{lstlisting}
$ python3 5.3.py
\end{lstlisting}
\begin{figure}[!h]
\centering
\includegraphics[width=\columnwidth]{./figs/5.3.png}
\caption{The Scatter Plot}
\label{fig:5.3}
\end{figure}
	\item Guess how to estimate $X$ from $Y$.\\
	\solution:
	\begin{enumerate}
	    \item If $Y<0$ then probably $X=-1$
	    \item If $Y>0$ then probably $X=1$
	\end{enumerate}
\label{ml-ch4_sim}
\item Find 
\begin{equation}
	P_{e|0} = \pr{\hat{X} = -1|X=1}
\end{equation}
and 
\begin{equation}
	P_{e|1} = \pr{\hat{X} = 1|X=-1}
\end{equation}
	\solution:
\begin{lstlisting}
wget https://github.com/GovindaRohith/Assignments/blob/main/Randomnum/codes/5.5.c
wget https://github.com/GovindaRohith/Assignments/blob/main/Randomnum/codes/source.h
\end{lstlisting}
Download the above files and execute the following commands to get the result
\begin{lstlisting}
$ gcc 5.5.c
$ ./a.out
\end{lstlisting}
\begin{align}
P_{e|0}=0.499033\\
P_{e|1}=0.500138
\end{align}
\item Find $P_e$ assuming that $X$ has equiprobable symbols.\\
\solution
\begin{align}
    P_e=P(X=1)P_{e|0}+P(X=-1)P_{e|1}
\end{align}
Since X is equiprobable
\begin{align}
    P(X=1)=P(X=-1)=0.5\\
    \implies P_e=\frac{P_{e|0}+P_{e|1}}{2}\\
    \implies \boxed{P_e=0.499585}
\end{align}
\item
Verify by plotting  the theoretical $P_e$ with respect to $A$ from 0 to 10 dB.  \\
\solution:
\begin{align}
    P_{e|0}&=\pr{\hat{X}=-1|X=1}\\
    P_{e|0}&=\pr{AX+N<0|X=1}\\
    P_{e|0}&=\pr{N<-A}\\
    P_{e|0}&=\int_{-\infty }^{-A}\frac{e^{\frac{-x^2}{2}}}{\sqrt{2\pi}}\\
    P_{e|0}&=\int_{A }^{\infty}\frac{e^{\frac{-x^2}{2}}}{\sqrt{2\pi}}\\
    P_{e|0}&=Q_N(A)\\
    \text{Similarly, }P_{e|1}&=Q_N(A)
\end{align}
\begin{lstlisting}
wget https://github.com/GovindaRohith/Assignments/blob/main/Randomnum/codes/5.6.py
\end{lstlisting}
Download the above files and execute the following commands to produce Fig.\ref{fig:5.6}
\begin{lstlisting}
$ python3 5.6.py
\end{lstlisting}
\begin{figure}[!h]
\centering
\includegraphics[width=\columnwidth]{./figs/5.6.png}
\caption{$P_e(A)$ with semilog-y axis}
\label{fig:5.6}
\end{figure}
\item Now, consider a threshold $\delta$  while estimating $X$ from $Y$. Find the value of $\delta$ that maximizes the theoretical $P_e$.
\\
\solution
To estimate $X$ from $Y$, we now consider the following:
\begin{align}
    X = 
    \begin{cases}
        1, & Y > \delta \\
        -1, & Y < \delta
    \end{cases}
\end{align}
Therefore,
\begin{align}
P_{e|0} &= \pr{\hat{X} = -1 | X = 1} \\
&= \pr{AX+N < \delta | X = 1} \\
\implies P_{e|0}&= \pr{N < \delta - A} \\
&= \int _{-\infty} ^{\delta - A} \frac{1}{\sqrt{2\pi}} e^{-\frac{x^2}{2}} dx \\
&= \int _{A - \delta} ^{\infty} \frac{1}{\sqrt{2\pi}} e^{-\frac{x^2}{2}} dx \\
\implies P_{e|0}&= Q_N(A - \delta) \\
\intertext{Similarly,}
P_{e|1} &= Q_N(A+\delta) \\
\label{eq:Pe_equation}
P_e &= P_{e|0} \pr{X = 1} + P_{e|1} \pr{X = -1} \\
&= \frac{Q_N(A - \delta) + Q_N(A + \delta)}{2} 
\end{align}
Differentiating the above equation wrt $\delta$:
\begin{align}
0 &= \frac{d}{d\delta} \left(\frac{Q_N(A - \delta) + Q_N(A + \delta)}{2}\right) \\
&= \frac{1}{2} \left(\frac{1}{\sqrt{2\pi}} e^{-\frac{(\delta - A)^2}{2}} - \frac{1}{\sqrt{2\pi}} e^{-\frac{(A + \delta)^2}{2}} \right) \\
\intertext{Therefore,}
(\delta - A)^2 &= (\delta + A)^2 \\
\implies \boxed{\delta = 0}
\end{align}
\item Repeat the above exercise when 
    \begin{align}
        p_{X}(0) = p
    \end{align}
\\
\solution
Using Eq. \eqref{eq:Pe_equation}, we have:
\begin{align}
    P_e &= P_{e|0} p+ P_{e|1} (1-p) \\
    &= p Q_N(A - \delta) + (1-p)Q_N(A + \delta)
\intertext{Differentiating as before, we get:}
0 &= p \frac{1}{\sqrt{2\pi}} e^{-\frac{(\delta - A)^2}{2}} - (1-p)\frac{1}{\sqrt{2\pi}} e^{-\frac{(A + \delta)^2}{2}} 
\end{align}
 \begin{align}
e^{\frac{{(\delta +A)}^2-{((\delta -A))}^2}{2}}=\frac{1-p}{p}\\
\implies \boxed{\delta=0}
 \end{align}
    \item Repeat the above exercise using the MAP criterion.
\\
\solution
Assume that $\pr{X = -1} = p$, and $\pr{X = 1} = (1-p)$.\\
From Total Probability Theorem,
we have:
\begin{align}
\nonumber p_Y(y) &= p_{Y|X = -1}(y|-1) \pr{X = -1} \\
&+ p_{Y| X = 1}(y|1) \pr{X = 1} \\
p_Y(y) &=  p \times p_{(-A + N)}(y) \\
&+ (1-p) \times p_{(A+N)} (y) 
\end{align}
 
where $p_Y(y)$ is the pdf of $Y$. Now, $p_{(-A + N)}$ is just the pdf of a shifted
normal distribution, and therefore:
\begin{align}
    p_Y(y) &=  p \frac{e^{-\frac{(y+A)^2}{2}}}{\sqrt{2\pi}} + \left(1-p\right) \frac{e^{-\frac{(y-A)^2}{2}}}{\sqrt{2\pi}}
\end{align}
To use the MAP criterion, we must find $p_{X|Y}(x|y)$. To do this, we use the Theorem of Conditional Probability:
\begin{align}
    p_{X|Y}(x|y) &= \frac{p_{Y|X}(y|x) \times p_X(x)}{p_Y(y)}
\end{align}
When $X=1$, we have:
\begin{align}
    p_{X|Y}(1|y) &= \frac{p_{Y|X}(y|1) \times p_X(1)}{p_Y(y)} \\
    &= \frac{\left(1-p\right) \frac{e^{-\frac{(y-A)^2}{2}}}{\sqrt{2\pi}}}{ p \frac{e^{-\frac{(y+A)^2}{2}}}{\sqrt{2\pi}} + \left(1-p\right) \frac{e^{-\frac{(y-A)^2}{2}}}{\sqrt{2\pi}}} \\
    &= \frac{\left(1-p\right) e^{2yA}}{p + \left(1-p\right) e^{2yA}}
\end{align}
Similarly, when $X = -1$, we get:
\begin{align}
    p_{X|Y}(-1|y) &= \frac{p}{p + \left(1-p\right) e^{2yA}} 
\end{align}
Therefore, when $ p_{X|Y}(1|y) >  p_{X|Y}(-1|y)$, we have:
\begin{align}
    \frac{\left(1-p\right) e^{2yA}}{p + \left(1-p\right) e^{2yA}} &> \frac{p}{p + \left(1-p\right) e^{2yA}} \\
    e^{2yA} &> \frac{p}{\left(1-p\right)} \\
    \label{eq:y-condition}
    y &> \frac{1}{2A} \ln{\frac{p}{\left(1-p\right)}}
\end{align}
Therefore, when Eq. \eqref{eq:y-condition}, we can assert that $X = 1$, and $X = -1$ otherwise.
Now, consider when $p = \frac{1}{2} $.
We have:
\begin{align}
    y &> \frac{1}{2A} \ln{\frac{p}{\left(1-p\right)}} \\
    &= \frac{1}{2A} \ln{1} \\
    &= 0
\end{align}
Therefore, when $y > 0$, we choose $X = 1$, and we choose $X = -1$ otherwise.


		\end{enumerate}
\section{Guasssian to Other}
\begin{enumerate}[label=\thesection.\arabic*
,ref=\thesection.\theenumi]
\item Let $X_1  \sim \gauss{0}{1} $ and
$X_2  \sim \gauss{0}{1} $.Plot the CDF and PDF of 
\begin{align}
    V=X_1^2+X_2^2
    \end{align}
\solution\\
The CDF and PDF of $V$ is plotted in Fig. \ref{fig:6.1cdf} and \ref{fig:6.1pdf} using the codes below
\begin{lstlisting}
wget https://github.com/GovindaRohith/Assignments/blob/main/Randomnum/codes/6.1cdf.py
wget https://github.com/GovindaRohith/Assignments/blob/main/Randomnum/codes/6.1pdf.py
\end{lstlisting}
Download the above files and execute the following commands to produce Fig.\ref{fig:6.1cdf} and \ref{fig:6.1pdf}
\begin{lstlisting}
$ python3 6.1cdf.py
$ python3 6.1pdf.py
\end{lstlisting}
\begin{figure}[!h]
\centering
\includegraphics[width=\columnwidth]{./figs/6.1cdf.png}
\caption{The CDF of $V$}
\label{fig:6.1cdf}
\end{figure}
\addtocounter{figure}{+1}
\begin{figure}[!h]
\centering
\includegraphics[width=\columnwidth]{./figs/6.1pdf.png}
\caption{The PDF of $V$}
\label{fig:6.1pdf}
\end{figure}
\item If
%
\begin{equation}
F_{V}(x) = 
\begin{cases}
1 - e^{-\alpha x} & x \geq 0 \\
0 & x < 0,
\end{cases}
\end{equation}
%
find $\alpha$.\\
\solution
Assuming $X_1$ and $X_2$ are i.i.d\\
\begin{align}
X_1 = r\cos \Theta  \\
X_2 = r\sin \Theta 
\end{align}
The Jacobian matrix transforming $r, \Theta$ to $X_1,X_2$  is defined as 
\begin{align}
	\vec{J} &= \myvec{
		\frac{\partial X_1}{\partial r}
		& \frac{\partial X_1}{\partial \Theta} \\
		\frac{\partial X_2}{\partial r}
		&\frac{\partial X_2}{\partial \Theta}
		}\\
		\implies \vec{J} &= \myvec{
		\cos{\Theta}
		& -r\sin{\Theta} \\
		\sin{\Theta}
		&r\cos{\Theta}
		}\\
		\implies \mydet{\vec{J}}&=r\\
		\end{align}
		\begin{align}
f_{X_1,X_2}(x_1,x_2)&=f_{X_1}(x_1)f_{X_2}(x_2)\\
\implies f_{X_1,X_2}(x_1,x_2)&=\frac{e^\frac{{-x_1^2-x_2^2}}{2}}{2\pi}\\
\implies f_{X_1,X_2}(x_1,x_2)&=\frac{e^\frac{{-r^2}}{2}}{2\pi}\\
f_{R, \Theta} \brak{r, \theta} &= 
	f_{X_1,X_2}\brak{x_1,x_2} \mydet{\vec{J}}
	\end{align}
	\begin{align}
	\implies f_{R, \Theta} \brak{r, \theta} &= \frac{re^\frac{{-r^2}}{2}}{2\pi}\\
	f_R(r) &= \int_{0}^{2\pi}
		f_{R, \Theta} \brak{r, \theta} \, d\theta\\
		f_R(r) &= \int_{0}^{2\pi}
		\frac{re^\frac{{-r^2}}{2}}{2\pi} \, d\theta\\
		\implies \boxed{f_R(r)=re^{\frac{-r^2}{2}}}\\
		    F_V(r)&=F_{X_1^2+X_2^2}(r)\\
		    \implies F_V(r)&=F_{R^2}(r)
		    \end{align}
		\begin{align}
		    \implies F_V(r)&=\Pr{(R^2\le x)}\\
		    \implies F_V(r)&=\Pr{(R\le \sqrt{x})}\\
		    \text{But }F_V(x)&=\Pr{(R\le x)}\\
		    \implies F_V(x)&=\int _{0}^{x}f_R(r)dr\\
F_V(x)&=1-e^{\frac{-x^2}{2}}\\
\implies F_V(r)&=\begin{cases}
0 & r<0\\
1-e^{\frac{-r}{2}} &r\ge 0\\
\end{cases}
		\end{align}
		\begin{align}
		    \implies \boxed{\alpha=\frac{1}{2}}
		\end{align}
\item Plot the CDF and PDF of 
\begin{align}
    A=\sqrt{V}
\end{align}
\solution: For $r\ge 1$
\begin{align}
	F_A(r) &= \pr{A \leq r} \\
	&= \pr{\sqrt{V} \leq r} \\
	&= \pr{V \leq r^2} \\
	&= F_V(r^2) = 1 - e^{-\frac{r^2}{2}}
\end{align}
and so on differentiating 
\begin{align}
	f_A(r) = re^{-\frac{r^2}{2}}
\end{align}
Thus, the CDF and PDF of $A$ is given by
\begin{align}
	F_V(r) = 
	\begin{cases}
		1 - e^{-\frac{r^2}{2}} & r \geq 0 \\
		0 & r < 0 
	\end{cases} \label{eq:ral-cdf} \\
	f_V(r) = 
	\begin{cases}
		re^{-\frac{r}{2}} & r \geq 0 \\
		0 & r < 0
	\end{cases} \label{eq:ral-pdf} 
\end{align}
The CDF and PDF of $A$ is plotted in Fig. \ref{fig:6.1cdf} and \ref{fig:6.1pdf} using the codes below
\begin{lstlisting}
wget https://github.com/GovindaRohith/Assignments/blob/main/Randomnum/codes/6.3cdf.py
wget https://github.com/GovindaRohith/Assignments/blob/main/Randomnum/codes/6.3pdf.py
\end{lstlisting}
Download the above files and execute the following commands to produce Fig.\ref{fig:6.3cdf} and \ref{fig:6.3pdf}
\begin{lstlisting}
$ python3 6.3cdf.py
$ python3 6.3pdf.py
\end{lstlisting}
\begin{figure}[!h]
\centering
\includegraphics[width=\columnwidth]{./figs/6.3cdf.png}
\caption{The CDF of $V$}
\label{fig:6.3cdf}
\end{figure}
\begin{figure}[!h]
\centering
\includegraphics[width=\columnwidth]{./figs/6.3pdf.png}
\caption{The PDF of $V$}
\label{fig:6.3pdf}
\end{figure}
\end{enumerate}
\end{enumerate}
\end{document}
