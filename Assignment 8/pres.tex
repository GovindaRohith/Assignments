%%%%%%%%%%%%%%%%%%%%%%%%%%%%%%%%%%%%%%%%%%%%%%%%%%%%%%%%%%%%%%%
%
% Welcome to Overleaf --- just edit your LaTeX on the left,
% and we'll compile it for you on the right. If you open the
% 'Share' menu, you can invite other users to edit at the same
% time. See www.overleaf.com/learn for more info. Enjoy!
%
%%%%%%%%%%%%%%%%%%%%%%%%%%%%%%%%%%%%%%%%%%%%%%%%%%%%%%%%%%%%%%%

% Inbuilt themes in beamer
\documentclass{beamer}

%packages:
% \usepackage{tfrupee}
% \usepackage{amsmath}
% \usepackage{amssymb}
% \usepackage{gensymb}
% \usepackage{txfonts}

% \def\inputGnumericTable{}

% \usepackage[latin1]{inputenc}                                 
% \usepackage{color}                                            
% \usepackage{array}                                            
% \usepackage{longtable}                                        
% \usepackage{calc}                                             
% \usepackage{multirow}                                         
% \usepackage{hhline}                                           
% \usepackage{ifthen}
% \usepackage{caption} 
% \captionsetup[table]{skip=3pt}  
% \providecommand{\pr}[1]{\ensuremath{\Pr\left(#1\right)}}
% \providecommand{\cbrak}[1]{\ensuremath{\left\{#1\right\}}}
% %\renewcommand{\thefigure}{\arabic{table}}
% \renewcommand{\thetable}{\arabic{table}}      

\setbeamertemplate{caption}[numbered]{}

\usepackage{enumitem}
\usepackage{tfrupee}
\usepackage{amsmath}
\usepackage{amssymb}
\usepackage{gensymb}
\usepackage{graphicx}
\usepackage{txfonts}

\def\inputGnumericTable{}

\usepackage[latin1]{inputenc}                                 
\usepackage{color}                                            
\usepackage{array}                                            
\usepackage{longtable}                                        
\usepackage{calc}                                             
\usepackage{multirow}                                         
\usepackage{hhline}                                           
\usepackage{ifthen}
\usepackage{caption} 
\captionsetup[table]{skip=3pt}  
\providecommand{\pr}[1]{\ensuremath{\Pr\left(#1\right)}}
\providecommand{\cbrak}[1]{\ensuremath{\left\{#1\right\}}}
\renewcommand{\thefigure}{\arabic{table}}
\renewcommand{\thetable}{\arabic{table}}   
\providecommand{\brak}[1]{\ensuremath{\left(#1\right)}}

% Theme choice:
\usetheme{CambridgeUS}

% Title page details: 
\title{Assignment 8} 
\author{Govinda Rohith Y}
\date{\today}
\logo{\large \LaTeX{}}


\begin{document}

% Title page frame
\begin{frame}
    \titlepage 
\end{frame}

% Remove logo from the next slides
\logo{}


% Outline frame
\begin{frame}{Outline}
    \tableofcontents
\end{frame}


% Lists frame
\section{Question}
\begin{frame}{Question}

\begin{block}{\textbf{Example 15 [NCERT 12] :}}
        A person has undertaken a construction job. The probabilities are 0.65 that there will be strike, 0.80 that the construction job will be completed on time if there is no strike, and 0.32 that the construction will be completed on time if there is a strike. Determine the probability that the construction job will be completed on time.\\
    \end{block}

\end{frame}


% Blocks frame
\section{Denote Random Variables}
\begin{frame}{Denote Random Variables}
    \begin{block}{Assign events to random variables}
    Denote the random variables $X,Y\in \{0,1\}$. Events for each random variable are displayed in  Table \ref{table:table1}:
    \end{block}
    \begin{table}[h!]
	\input{tables/table1}
	\caption{}
    \label{table:table1}
    \end{table}
     
     \end{frame} 
     \section{Given data}
     \begin{frame}{Given data}
\begin{block}{Represent the given data}
The probability we are required to find is $\pr{Y=1}$. Various values of probability are displayed in Table \ref{table:table2}:

\end{block}
\begin{table}[h!]
	\input{tables/table2}
	\caption{}
    \label{table:table2}
    \end{table}
\end{frame} 

\section{Solution}
     \begin{frame}{Solution}
      From Total probability theorem
    \begin{align}
    \pr{Y=1}=\sum_{j=0}^1 \pr{X=j}\times\pr{Y=1|X=j} \\
    \implies\boxed{\pr{Y=1}=0.488}
\end{align}
     \end{frame}
\end{document}