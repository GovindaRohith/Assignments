%%%%%%%%%%%%%%%%%%%%%%%%%%%%%%%%%%%%%%%%%%%%%%%%%%%%%%%%%%%%%%%
%
% Welcome to Overleaf --- just edit your LaTeX on the left,
% and we'll compile it for you on the right. If you open the
% 'Share' menu, you can invite other users to edit at the same
% time. See www.overleaf.com/learn for more info. Enjoy!
%
%%%%%%%%%%%%%%%%%%%%%%%%%%%%%%%%%%%%%%%%%%%%%%%%%%%%%%%%%%%%%%%

% Inbuilt themes in beamer
\documentclass{beamer}

%packages:
% \usepackage{tfrupee}
% \usepackage{amsmath}
% \usepackage{amssymb}
% \usepackage{gensymb}
% \usepackage{txfonts}

% \def\inputGnumericTable{}

% \usepackage[latin1]{inputenc}                                 
% \usepackage{color}                                            
% \usepackage{array}                                            
% \usepackage{longtable}                                        
% \usepackage{calc}                                             
% \usepackage{multirow}                                         
% \usepackage{hhline}                                           
% \usepackage{ifthen}
% \usepackage{caption} 
% \captionsetup[table]{skip=3pt}  
% \providecommand{\pr}[1]{\ensuremath{\Pr\left(#1\right)}}
% \providecommand{\cbrak}[1]{\ensuremath{\left\{#1\right\}}}
% %\renewcommand{\thefigure}{\arabic{table}}
% \renewcommand{\thetable}{\arabic{table}}      

\setbeamertemplate{caption}[numbered]{}

\usepackage{enumitem}
\usepackage{tfrupee}
\usepackage{amsmath}
\usepackage{amssymb}
\usepackage{gensymb}
\usepackage{graphicx}
\usepackage{txfonts}

\def\inputGnumericTable{}

\usepackage[latin1]{inputenc}                                 
\usepackage{color}                                            
\usepackage{array}                                            
\usepackage{longtable}                                        
\usepackage{calc}                                             
\usepackage{multirow}                                         
\usepackage{hhline}                                           
\usepackage{ifthen}
\usepackage{caption} 
\captionsetup[table]{skip=3pt}  
\providecommand{\pr}[1]{\ensuremath{\Pr\left(#1\right)}}
\providecommand{\cbrak}[1]{\ensuremath{\left\{#1\right\}}}
\renewcommand{\thefigure}{\arabic{table}}
\renewcommand{\thetable}{\arabic{table}}   
\providecommand{\brak}[1]{\ensuremath{\left(#1\right)}}

% Theme choice:
\usetheme{CambridgeUS}

% Title page details: 
\title{Assignment 9} 
\author{Govinda Rohith Y}
\date{\today}
\logo{\large \LaTeX{}}


\begin{document}

% Title page frame
\begin{frame}
    \titlepage 
\end{frame}

% Remove logo from the next slides
\logo{}


% Outline frame
\begin{frame}{Outline}
    \tableofcontents
\end{frame}


% Lists frame
\section{Question}
\begin{frame}{Question}

\begin{block}{\textbf{2-24(Papoullis):}}
        Box 1 contains 1000 bulbs of which $10\%$ are defective. Box 2 contains 2000 bulbs which $5\%$ are defective. Two bulbs are picked from a randomly selected box.
\begin{enumerate}[label=(\alph*)]
    \item Find the probability that both bulbs are defective.
    \item Assuming that both are defective, find the probability that they came from box 1.
\end{enumerate}
    \end{block}

\end{frame}


% Blocks frame
\section{Denote Random Variables}
\begin{frame}{Denote Random Variables}
    \begin{block}{Assign events to random variables}
   Denote the random variables $Y\in \cbrak{0,1}$. and $X\in \cbrak{0,1,2}$ Events are described in Table \ref{table:table1}:
    \end{block}
    \begin{table}[h!]
	\input{tables/table1}
	\caption{}
    \label{table:table1}
    \end{table}
     
     \end{frame} 
     \section{Given data}
     \begin{frame}{Given data}
\begin{block}{Represent the given data}

\end{block}
\begin{table}[h!]
	\input{tables/table2}
	\caption{}
    \label{table:table2}
    \end{table}
\end{frame} 

\section{Solution(a)}
     \begin{frame}{Solution(a)}
     \begin{block}{Solution}
     $\pr{X=0}$ denotes the probability that both bulbs are defective.\\
     \end{block}
      From Total probability theorem
        \begin{align}
        \pr{X=0}=\sum_{i=0}^1 \pr{Y=i}\pr{X=0|Y=i}\\
            \implies \boxed{\pr{X}\approx 0.006193}
        \end{align}
     \end{frame}
     
     
     \section{Solution(b)}
     \begin{frame}{Solution(b)}
     \begin{block}{Solution}
     $\pr{Y=0|X=0}$ denotes two bulbs are picked from box 1 assuming both are defective.
     \end{block}
     From Bayes theorem
        \begin{align}
            \pr{Y=0|X=0}=\frac{\pr{Y=0}\pr{X=0|Y=0}}{\pr{X}}\\
            \implies \boxed{\pr{Y=0|X=0}\approx0.8}
        \end{align}
     \end{frame}
\end{document}
